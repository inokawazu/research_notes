\section{Math Derivations and References}

\subsection{Conventions and Results}

\paragraph{Gauge Theory Objects}
For a gauge theory of group, $G$, the field strength two-form is written as
%
\begin{equation}
  F = dA + i A\wedge A \equiv dA + i \frac 12 [A\wedge A]
\end{equation}
%
in the physicist's (Hermitian) convention.
$A$ is the gauge connection.

\paragraph{Derivations}
The external covariant derivative is of some $r$-form is given as
%
\begin{equation}
  \cd \omega_r := d\omega_r + i [\omega_r \wedge A] = d\omega_r + i \omega_r A  - (-1)^{r} i A \omega_r
\end{equation}
%
The dual derivation is given as $\cd^\dagger := \star^{-1}\cd\star$.
It has the property of $(\omega_r, \cd \xi_{r-1}) = (\cd^\dagger\omega_r, \xi_{r-1})$.
For the inner product of $r$-forms $(\cdot, \cdot)$.

\paragraph{Epsilon Tensor}
The epsilon tensor is used for the volume form, $dV \equiv \frac 1{n!}\varepsilon_I dx^I$ where $n$ is the number of dimensions.
$\varepsilon_I = N(I)$ where $N(I)$ is the parity of $I$ if $I$ was taken as a permutation.
The raised index version, $\varepsilon^I$, can be related to lowered index version as $\varepsilon^I = \det(g)\varepsilon_I$.
This leads to the identity $\frac {1}{r!}\varepsilon^{K_r I} \varepsilon_{K J} = \det g \delta^I_J$.
$\delta$ is the generalized Kronecker delta.

\paragraph{Hodge Dual} The Hodge dual star operators is defined as 
%
\begin{equation}
  \star \omega_r = {\frac {\sqrt{-g}}{r!(n-r)!}} (\omega_r)_{I} \varepsilon^I_{~J} dx^{J}
\end{equation}
%
$\omega_r$ is an $r$-form.

\paragraph{Implicit Wedge} The wedge product is implicit if not stated otherwise.

\paragraph{Yang-Mills Action}
We write a free Yang-Mills (YM) action as
%
\begin{equation}
  S_{YM} := -\kappa \int \frac 12\tr F \wedge \star F
\end{equation}
%
The equation of motion is given as
%
\begin{equation}
  \kappa\star\cd \star F = 0
\end{equation}
%
In index form we have
%
\begin{equation}
  \frac 1{\sqrt{-g}} \cd_a (\sqrt{-g} F^{ab}) = 0
\end{equation}
%
$\cd_a$ here is the covariant derivative $\cd_a = \nabla_a + i A_a$

%%%%%%%%%%%%%%%%%%%%%%%%%%%%%%%%%%%%%%%%%%%%%%%%%%%%%%%%%%%%%%%%%%%%%%%%%%%%%%%%%%%
\subsection{Derivations}

\paragraph{Hodge Star Inverse} We will derive the Hodge Star inverse.
%
\begin{equation}
  \begin{aligned}
    \star \star \omega_r &= \star \left( {\frac {r!\sqrt{-g}}{(n-r)!}} (\omega_r)_{I} \varepsilon^I_{~J} dx^{J} \right)\\
                         &= \left( \frac{\sqrt{-g}}{r!(n-r)!} (\omega_r)_{I} \varepsilon^I_{~J}  \right) \frac{\sqrt{-g}}{r!} \varepsilon^J_{~K} dx^K\\ 
                         &= \frac{-g}{r!(n-r)!r!} (\omega_r)_{I} \varepsilon^I_{~J}  \varepsilon^J_{~K} dx^K\\ 
                         &= \frac{-g}{r!(n-r)!r!} (\omega_r)_{I} G^{II'} \varepsilon_{I'J} \varepsilon^{JK'} G_{K'K} dx^K\\ 
                         &= \frac{-g}{r!(n-r)!r!} (\omega_r)_{I} G^{II'} \left(\frac {(n-r)!}g (-1)^{r(n-r)} \delta_{I'}^{K'}\right) G_{K'K} dx^K\\ 
                         &= (-1)^{r(n-r) + 1} \frac 1{r!}(\omega_r)_I dx^I\\ 
  \end{aligned}
\end{equation}
%
So, this semi-involution operations implies the form of the inverse is
%
\begin{equation}
  \star^{-1} \equiv (-1)^{r(n-r) + 1} \star
\end{equation}

\paragraph{Hodge Star Inner Product}
The hodge star can be used to create an inner product on $r$-forms, $\omega_r$, $\xi_r$.
The inner product is defined as
%
\begin{equation}
  (\omega_r, \xi_r) := \int_{\mathcal M} \omega_r\wedge\star\xi_r
\end{equation}
%
\begin{equation}
  \begin{aligned}
    \omega_r \star \xi_r &= \frac{\sqrt{-g}}{(n-r)!r!r!} (\omega_r)_K (\xi_r)_I \varepsilon^I_{~J} dx^{KJ}\\
                            &= \frac{\sqrt{-g}}{(n-r)!r!r!}  ((n-r)!r!) (\omega_r)_K (\xi_r)^K dx^{1\ldots n}\\
                            &= \sqrt{-g} (\omega_r)_K (\xi_r)^K dx^{1\ldots n}\\
  \end{aligned}
\end{equation}

\paragraph{Equation of Motion of Free YM}
The variation of the free YM action gives the following.
%
\begin{equation}
  \delta S_{YM} = \kappa \int \tr \delta F \wedge \star F
\end{equation}
%
We can isolate the variation $\delta A$.
%
\begin{equation}
  \begin{aligned}
    \delta S_{YM} &= -\kappa \int \tr \delta F \wedge \star F\\
                  &= -\kappa \int \tr (d\delta A + i \delta A\wedge A + i A\wedge \delta A) \wedge \star F\\
  \end{aligned}
\end{equation}

We can handle these terms one at a time
%
\begin{equation}
  \begin{aligned}
    \delta S_{YM} &\supset -\kappa \int \tr d\delta A \wedge \star F\\
                  &= -\kappa \int \tr (d( \delta A \wedge \star F ) + \delta A \wedge d\star F)\\
                  &= -\kappa \int \tr (\delta A \wedge d\star F)\\
  \end{aligned}
\end{equation}
%
\begin{equation}
  \begin{aligned}
    \delta S_{YM} &\supset -\kappa \int \tr(i \delta A\wedge A + i A\wedge \delta A) \wedge \star F\\
                  &= -\kappa \int \tr(i \delta A\wedge A \wedge \star F + i A\wedge \delta A  \wedge \star F)\\
                  &= -\kappa \int \tr(i \delta A\wedge A \wedge \star F + i \delta A  \wedge \star F \wedge A)\\
                  &= -\kappa \int \tr( \delta A \wedge (i A \wedge \star F + i \star F \wedge A) )\\
  \end{aligned}
\end{equation}

Together we have

\begin{equation}
  \begin{aligned}
    \delta S_{YM} &= -\kappa \int \tr ( \delta A \wedge (\delta A \wedge d\star F + i A \wedge \star F + i \star F \wedge A) )\\
                  &= -\kappa \int \tr ( \delta A \wedge \cd ( \star F ) )\\
                  &= -\kappa \int \tr ( \delta A \wedge \star (\star^{-1}\cd ( \star F )) )\\
                  &= -\kappa \int \tr ( \delta A \wedge \star (-\star\cd \star F) )\\
                  &=  \int \tr ( \delta A \wedge \star (\kappa\star\cd \star F) )\\
  \end{aligned}
\end{equation}

We can now calculate $\star \cd \star \omega_r$ in index notation.

\begin{equation}
  \begin{aligned}
    \star \cd \star \omega_r &= \star \cd \star \frac 1{r!}(\omega_r)_I dx^I\\
                                    &= \star \cd \left( \frac{\sqrt{-g}}{r!(n-r)!} (\omega_r)_{I} \varepsilon^I_{~J} dx^J \right)\\
                                    &= \star \left( \frac{1}{r!(n-r)!} \cd_k ( \sqrt{-g}(\omega_r)_{I}  \varepsilon^I_{~J} )  dx^{kJ}\right)\\
                                    &=  \frac{1}{r!(n-r)!} \cd_k ( \sqrt{-g}(\omega_r)_{I}  \varepsilon^I_{~J} )  ( \varepsilon^{kJ}_{~~~L}\frac{\sqrt{-g}}{ (r-1)! } dx^L )\\
                                    &=  \frac{\sqrt{-g}}{r!(r-1)!(n-r)!} \cd_k ( \sqrt{-g}(\omega_r)_{I}  \varepsilon^I_{~J} ) \varepsilon^{kJ}_{~~~L} dx^L\\
                                    &= \frac{\sqrt{-g}}{r!(r-1)!(n-r)!} \cd_k ( \sqrt{-g}(\omega_r)_{I}  G^{II'} ) \varepsilon_{I'J} \varepsilon^{kJ}_{~~~L} dx^L\\
                                    &= \frac{\sqrt{-g}}{r!(r-1)!(n-r)!} \cd_k ( \sqrt{-g}(\omega_r)_{I}  G^{II'} ) \varepsilon_{I'J} \varepsilon^{kJL'} G_{L'L} dx^L\\
                                    &= \frac{\sqrt{-g}}{r!(r-1)!(n-r)!} \cd_k ( \sqrt{-g}(\omega_r)_{I}  G^{II'} ) (-1)^{(r-1)(n-r)} \frac{ (n-r)! }{g}\delta^{kL'}_{I'} G_{L'L} dx^L\\
                                    &= \frac{(-1)^{(r-1)(n-r)+1}}{\sqrt{-g}}\frac{1}{r!(r-1)!}  \delta^{kL'}_{I'} G_{L'L} \cd_k ( \sqrt{-g}(\omega_r)_{I}  G^{II'} )  dx^L\\
                                    &= \frac{(-1)^{(r-1)(n-r)+1}}{\sqrt{-g}}\frac{1}{(r-1)!} G_{L'L} \cd_k ( \sqrt{-g}(\omega_r)_{I}  G^{IkL'} )  dx^L\\
                                    &= \frac{(-1)^{(r-1)(n-r)+1}}{\sqrt{-g}}\frac{1}{(r-1)!}  \cd_k ( \sqrt{-g}(\omega_r)^{kL'} G_{L'L})  dx^L\\
  \end{aligned}
\end{equation}
%
therefore
%
\begin{equation}
  \star \cd \star F = \frac{1}{\sqrt{-g}}  \cd_k ( \sqrt{-g}F^{k}_{~a})dx^a
\end{equation}
%
or
%
\begin{equation}
  \flat (\star \cd \star F) = \frac{1}{\sqrt{-g}}  \cd_k ( \sqrt{-g}F^{ka})\pd_a
\end{equation}

\paragraph{5D Chern-Simons Action Variation} 
Now let's calculate the variation of the action.
%
\begin{equation}
  \delta S_\mathrm{CS} = \frac{N_c}{24\pi^2} \int \omega_5(\delta A)
\end{equation}
%
We can handle each term of the $\omega_5$ individually.

\begin{equation}
  \begin{aligned}
    \delta S_\mathrm{CS} &\supset \frac{N_c}{24\pi^2} \int \tr(\delta (A F^2))\\
    \delta S_\mathrm{CS} &\supset \frac{N_c}{24\pi^2} \int \tr(\delta A F^2 + A \delta F F + A F\delta F)\\
    \delta S_\mathrm{CS} &\supset \frac{N_c}{24\pi^2} \int \tr(\delta A F^2 + A (d\delta A + i \delta A A + i A \delta A) F + A F(d\delta A + i \delta A A + i A \delta A))\\
    \delta S_\mathrm{CS} &\supset \frac{N_c}{24\pi^2} \int \tr(\delta A (F^2 + i (A^2 F + 2 A F A + F A^2 )) +  d\delta A (F A + A F))\\
    \delta S_\mathrm{CS} &\supset \frac{N_c}{24\pi^2} \int \tr(\delta A (F^2 + i (A^2 F + 2 A F A + F A^2 ) + d(F A + A F)))\\
    \delta S_\mathrm{CS} &\supset \frac{N_c}{24\pi^2} \int \tr(\delta A (3 F^2 + i (A^2 F + F A^2 ) ))\\
  \end{aligned}
\end{equation}

\begin{equation}
  \begin{aligned}
    \delta S_\mathrm{CS} &\supset \frac{N_c}{24\pi^2} \int \tr\left(\delta (\frac i2 A^3 F)\right)\\
    \delta S_\mathrm{CS} &\supset \frac{N_c}{24\pi^2} \int \tr\left(\frac i2 ( \delta A A A F +  A \delta A A F + A A \delta A F + A^3 \delta F )\right)\\
    \delta S_\mathrm{CS} &\supset \frac{N_c}{24\pi^2} \int \tr\left(\frac i2 ( \delta A A A F +  A \delta A A F + A A \delta A F + A^3 \delta dA F + i A^3 \delta A A + i A^4 \delta A )\right)\\
    \delta S_\mathrm{CS} &\supset \frac{N_c}{24\pi^2} \int \tr\left(\frac i2 \delta A ( A^2 F +  A F A  + F A^2  + 2i A^4 + d(A^3)) \right)\\
  \end{aligned}
\end{equation}

\begin{equation}
  \begin{aligned}
    \delta S_\mathrm{CS} &\supset \frac{N_c}{24\pi^2} \int \tr\left(\delta \left(-\frac 1{10} A^5\right)\right)\\
    \delta S_\mathrm{CS} &\supset \frac{N_c}{24\pi^2} \int \tr\left(-\frac 1{10} \delta A (5 A^4)\right)\\
  \end{aligned}
\end{equation}

$$
\tr\left(A F^2 + \frac {i}2 A^3 F - \frac 1{10} A^5 \right)
$$

Adding all the terms together we get

\begin{equation}
  \begin{aligned}
    \delta S_\mathrm{CS} &= \frac{N_c}{24\pi^2} \int \tr\left(\delta A \left(
        3 F^2 +  iA^2 F + iF A^2  + 
        \frac i2A^2 F +  \frac i2A F A  + \frac i2F A^2  - A^4 + \frac i2 d(A^3) -\frac 12 A^4
    \right)
  \right)\\
    \delta S_\mathrm{CS} &= \frac{N_c}{24\pi^2} \int \tr\left(\delta A \left(
        3 F^2 +  i\frac 32 A^2 F + i \frac 32 F A^2
         +  \frac i2A F A + \frac i2 (FA^2-AFA+A^2F -i A^4) -\frac 32 A^4
    \right)
  \right)\\
    \delta S_\mathrm{CS} &= \frac{N_c}{24\pi^2} \int \tr\left(\delta A \left(
        3 F^2 +  2i(A^2 F + F A^2) - A^4
    \right)
  \right)\\
    \delta S_\mathrm{CS} &= \frac{N_c}{24\pi^2} \int \tr\left(\delta A \wedge \star 
      (-\star \left( 3 F^2 +  2i(A^2 F + F A^2) - A^4 \right))
  \right)\\
  \end{aligned}
\end{equation}

For Abelian generators\ldots
%
\begin{equation}
  \begin{aligned}
    \delta S_\mathrm{CS} &\supset \frac{N_c}{24\pi^2} \int \tr(\delta A \wedge \star (-3 \star F^2))\\
  \end{aligned}
\end{equation}
%
\begin{equation}
  \begin{aligned}
    \star F^2 &= \star (\frac 14 F_I F_J dx^{IJ})\\
    \star F^2 &= \frac {\sqrt{-g}}4 F_I F_J \varepsilon^{IJ}_{~~a} dx^a\\
  \end{aligned}
\end{equation}
